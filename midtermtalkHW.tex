\documentclass[12pt]{article}
\usepackage{listings}
\usepackage[colorlinks=true,pagebackref,linkcolor=blue]{hyperref}
\textwidth=7in
\textheight=9.5in
\topmargin=-1in
\headheight=0in
\headsep=.5in
\hoffset  -.85in

\lstset{
basicstyle=\footnotesize\ttfamily,
language=bash,
upquote=true,
breakatwhitespace=true,
columns=fullflexible,
keepspaces,
%numbers=none,
tabsize=3,
frame=blrt,
framextopmargin=5pt,
showstringspaces=false,
extendedchars=true
}

\pagestyle{empty}

\renewcommand{\thefootnote}{\fnsymbol{footnote}}

\begin{document}



\begin{center}
{\bf AMS 550.400 \quad Midterm Presentation Assignment \quad  Due Date:  {Wed,
    Oct 17}}\\
\vskip.2in
{\footnotesize Last Compiled on \today}
\end{center}

\setlength{\unitlength}{1in}

\begin{picture}(6,.1) 
\put(0,0) {\line(1,0){6.25}}         
\end{picture}

\renewcommand{\arraystretch}{2}


\begin{center}
\bf{General Turn-in Instruction} 
\end{center}
To complete this homework set, you are required to do the followings. 
Your work must be typed in \LaTeX\ using the course presentation template.  
The progression of your midterm presentation is to be
``recorded'' by making a git folder specifically for this midterm
presentation homework set. 
The burden of proof is on you, and if your git commit history
is sparse, then you may be liable for a penalty.  
A paper copy of the PDF output of your \LaTeX\ file is 
to be submitted to your instructor in class on the due date.
\emph{After} submitting the paper copy, but \emph{before} the end of
the due date, you will upload your work to your github by making a remote repository
specifically for the homework, and post the link to the repository 
at the designated \emph{Discussion} forum in Blackboard by making 
a thread just for you.  The repository name in your github should be
\texttt{550400.workstatement.final} and the discussion forum thread should
be named \texttt{YourFirstNameMiddleInitialLastName}, e.g.,
\texttt{BaracHObama} and \texttt{WillardMRommey}. 
You have till the end of the due date to finalize your github repository.  
However, any commit made after the class time of the due date will be 
inadmissible.  Finally, upload your video file to \emph{your}
Discussion Forum thread, i.e., \texttt{YourFirstNameMiddleInitialLastName}.  
\emph{Your attention to details in following this instruction will be 
critical, and if not followed exactly at the time of collection, the
homework set may be graded at $90\%$ of the full score}.

\vskip.25in

\begin{center}
{\bf  Midterm Presentation\footnote{Some of these materials are
    from \cite{RIPS2012}.} }
\end{center}

This major component of this course is training and practice in
giving good talks, presenting orderly information in a short
time. It's an art form requiring practice and discipline. You will
want to be good at giving a talk by the time of your midterm
presentation.  Public speaking is, for many, an intimidating experience at first, so
it helps to have an orderly process for getting used to it. Here are
some suggestions on how to prepare a presentation. 
The secret is repetition. The idea is to hold regular
practice sessions within your project, paying attention to timing and
structure. And don't worry about it if initially you mumble and
fumble. Practice makes perfect!

Your particular assignment in this homework set is to prepare a 20 min 
talk about the project that  you wrote in the work statement. This has two 
parts: 

\begin{itemize}
\item making beamer/LaTeX presentation slides 
\item making a presentation video of giving a talk
\end{itemize}

By making a video, I mean that you have a movie
file in which the screen shows the slide that you 
are currently talking about, and in the background,
your voice explain what is on the slide. You do 
not need to make it so that yourself actually 
appear in the movie but I do not stop you from 
doing so. 

You should make your recording as good as possible,
and try multiple times before finalizing the recording. 
How should you make a video recording? 
For OSX, QuickTime Player does make a 
simultaneous recording of voice and screen, and it 
comes with OSX by default. You can find it under 
the usual Applications folder.  
Unfortunately, for Windows, QuickTime Player does not 
have the same functionalities as the OSX version. However, 
there are ample free softwares that satisfies your needs. 
For example, check out this (\url{http://camstudio.org}).


The time requirement is strict in a sense that 
you should not go over 20 minutes. Making 
it exactly 20 minutes is hard. To give you 
what is acceptable, I would say 18 or 19 minutes 
are okay if your presentation naturally breaks there,
but 15 minutes is too short in any case. 
If you find yourself that you do not have enough 
materials to talk about, then that probably means 
that you have not given your project a serious 
consideration.

The contents of the talk should be a presentation
version of the work statement and any preliminary 
progress you have made in your project. A template for formal
presentations is shown next. Don't go overboard
in preparing colorfully dynamic and impressive graphics--except to the
extent that the graphics help directly to explain the technical
material and purpose of the project. 
\begin{itemize}
\item INTRODUCTION (5 minutes, approximate) (crystallized
  introduction, clarity for average listener)
\begin{itemize}
\item Title, sponsor identification, participants
\item Sponsor's business, relevance to problem area
\item Description/explanation of problem
\item List of deliverables
\end{itemize}
\item HEART OF TALK (10 minutes, approximate)
\begin{itemize}
\item Team's approach, in descriptive non-specialist language
\item Research accomplished -- discuss analysis/results -- oriented toward
specialists
\end{itemize}
\item CONCLUSION (5 minutes, approximate)
\begin{itemize}
\item Check against list of deliverables
\item Discuss work remaining to be done, negative results
\item Recommendations for future research
\end{itemize}
\end{itemize}

For the style guide for your presentation, please see 
Chapter 15 (Speaking about Multivariate Analysis) of \cite{WMA2005}.

\bibliographystyle{plain}
\nocite{*}
\bibliography{biblioMP}

\end{document}
