
\documentclass[12pt]{article}
\usepackage{listings}
\usepackage[colorlinks=true,pagebackref,linkcolor=blue]{hyperref}
\textwidth=7in
\textheight=9.5in
\topmargin=-1in
\headheight=0in
\headsep=.5in
\hoffset  -.85in

\lstset{
basicstyle=\footnotesize\ttfamily,
language=bash,
upquote=true,
breakatwhitespace=true,
columns=fullflexible,
keepspaces,
%numbers=none,
tabsize=3,
frame=blrt,
framextopmargin=5pt,
showstringspaces=false,
extendedchars=true
}

\pagestyle{empty}

\renewcommand{\thefootnote}{\fnsymbol{footnote}}

\begin{document}



\begin{center}
{\bf AMS 550.400 \quad Work Statement Revision Assignment \quad  Due Date:  {Wed,
    Oct 17}}\\
\vskip.2in
{\footnotesize Last Compiled on \today}
\end{center}

\setlength{\unitlength}{1in}

\begin{picture}(6,.1) 
\put(0,0) {\line(1,0){6.25}}         
\end{picture}

\renewcommand{\arraystretch}{2}


\begin{center}
\bf{General Turn-in Instruction} 
\end{center}
To complete this homework set, you are required to do the followings. 
Your work must be typed in \LaTeX\ using the course work statement
template.  
The progression of your work statement revision is to be
``recorded'' by making a git folder specifically for this revision
homework set. The burden of proof is on you, and if your git commit history
is sparse, then you may be liable for a penalty.  
A paper copy of the PDF output of your \LaTeX\ file is 
to be submitted to your instructor in class on the due date.
The \emph{original} marked copy of your first draft \emph{must} be
attached at the end of your \emph{final} draft.
Should you want to keep a record of the marked copy, please make a photocopy 
of the original marked draft for your reference.  
\emph{After} submitting the paper copy, but \emph{before} the end of
the due date, you will upload your work to your github by making a remote repository
specifically for the homework, and post the link to the repository 
at the designated \emph{Discussion} forum in Blackboard by making 
a thread just for you.  The repository name in your github should be
\texttt{550400.workstatement.final} and the discussion forum thread should
be named \texttt{YourFirstNameMiddleInitialLastName}, e.g.,
\texttt{BaracHObama} and \texttt{WillardMRommey}. 
You have till the end of the due date to finalize your github repository.  
However, any commit made after the class time of the due date will be 
inadmissible. \emph{Your attention to details in following this instruction will be 
critical, and if not followed exactly at the time of collection, the
homework set may be graded at $90\%$ of the full score}.

\vskip.25in

\begin{center}
{\bf Checklist for Revision}
\end{center}

\begin{itemize}
\item Make sure that your \emph{exogoneous} and \emph{endogenous}
variables are introduced in your \emph{Problem Statement} section.  Do
not refer anything directly as 
exogenous and endogenous, but anyone who knows
those technical words, \emph{should be able to} pick them out 
by reading \emph{Problem Statement} section.   See Chapter $1$ of
\cite{IMM1978} for review of these terminologies.  
\item Give appropriate credits to the publications that you are
  relying on.  Use the BibTeX.  For examples of a bibtex entry, see
  the file \texttt{biblioWS.bib} and for necessary \LaTeX\ codes needed
  in your workstatement \LaTeX\ file, see the very end of
  \texttt{workstatement.tex}  in the course repository.  The following
  webpage gives a nice tutorial on {BibTeX}:
\begin{center}
\url{http://en.wikibooks.org/wiki/LaTeX/Bibliography_Management}
\end{center}
\item In your \emph{Deliverable} section, if your project needs
  data/required assumptions/equipments  from your sponsoring
  organization, then make sure that you specify  the due date for
  the sponsor to deliver them, and specify your contingency plan
  should they fail to meet the deadline.
\item Do not leave any technical term unexplained.  For the
  guideline, see the text between Page $20$ and Page $25$ in \cite{WMA2005}.  
\item Explain your data and (proposed) method in \emph{Approach}
  section. For the guideline, see Chapter $12$ in \cite{WMA2005}.  
\item For each figure or table in the work statement, make sure to
  have a caption that addresses all four Ws (i.e., Who, What, When
  and Where).   See the text between Page 13 and page 15 in Chapter 2
  of \cite{WMA2005}.  
\item Do not be colloquial in your text.  Assume that your audience is a mixture of
  both statistically trained readers and professionals who are interested primarily in the answers rather than the technical details.  For some guideline, see the text
  between Page 380 and Page 385 in Chapter 16 of \cite{WMA2005}.
\end{itemize}

\vskip.25in

\begin{center}
{\bf Summary of Lectures\footnote{The most of these materials are  from \cite{RIPS2012}.} }
\end{center}
\paragraph{Introduction} 
Your \emph{Work Statement} is a key document that sets forth your
understanding of your research project at the start of the consulation
work, and it is important that you prepare it well. The initial draft will be
written by your team, then negotiated (and possibly modified by you in
negotiation) with your \emph{sponsor}\footnote{The term sponsor in
  this document refers to either your sponsoring organization or a
  person empowered to make agreements on behalf of your sponsoring
  organization, depending on context. Be sure to ascertain who can
  serve in that role.}.
When you have completed your Work
Statement in its final form and submitted it to your sponsor, it may
be negotiated and modified. But once agreed upon, your Work Statement
will serve as the foundation for your project.

But we all know that research is unpredictable, so you will want to
allow for contingency and be reasonable in your negotiation. This note
describes the content and structure of a Work Statement,
and offers suggestions for preparing the Work Statement and discussing 
it with your sponsor.

\paragraph{The Work Statement}
The Work Statement (referred to in government contracts as SOW or
Statement of Work) will be your team's commitment to the sponsor for
work agreed to.  In the reverse direction, it will include a statement
of what, if anything, you will receive from the sponsor, such as
software, data, hardware, or written materials, and it will serve as a
commitment by the sponsor not to demand more than is agreed to.

The Work Statement is an important contract between your team and the
sponsor, so it is important to prepare well and conduct your
negotiations carefully. Never use in your Work Statement technical
terms that you do not understand; otherwise you may discover too late
that you promised to do the impossible. Make sure you understand your
Work Statement very well before signing off on it.

A project Work Statement needn't be long; five to six pages would be
long. You just need to spell out what your team agrees to do and, in
turn, what you need to receive from the sponsor and when you need to
receive it, in order to get the work done.

Usually, these will include things like the following. 
\begin{itemize}
\item Problem statement
\item Background
\item Goal of your project (major direction you see the work aimed at,
  not necessarily what you bid to do)
\item Objectives (specific aims of your project, and schedule of
  results you expect to achieve)
\item Tasks (the things you will do to achieve your objectives) and a list of items
required from your sponsor in order to perform your tasks
\item Milestones (major checkpoints your team will use to stay on track)
\item Deliverables (specific work products you will deliver to the sponsor)
\item Schedule (dates for completing milestones and tasks and for deliverables)
\end{itemize}
Your Work Statement need not use the terms introduced above, but it
should address the major points they refer to.  And your Work Statement
need not be as fine-grained as what is implied by the terminology. 
For example, your team might set internal milestones as a
way of maintaining pace and coordination but not include them, or
include fewer of them, in the Work Statement. The Work Statement is
like a recipe -- getting the right ingredients and the right amount of
each ingredient is an art.

If your project does require the sponsor to provide something of
importance, be sure you include it in your Work Statement and specify
a last acceptable date for delivery of the material and contingency
plans that will allow you to proceed in the event of a failed
delivery. Note that this can include, for example, consultation to be
provided by the sponsor on the use of special equipment or software.

If your sponsor demands more than you think you can commit to doing,
try using phrases like, ``time permitting, we will attempt to do X'', or
``if our research leads successfully to A, we will then proceed to
investigating B.'' Statements like this show that you are aware of
where your sponsor wants to go and that you are committed to trying to
get there, but it also serves as fair notice that you believe the
sponsor may be asking too much to insist on B.

Your Work Statement is not just for experts. It will be of interest to
several parties of differing backgrounds and knowledge, not only to
your sponsor’s mentor and yourselves. For example, the manager(s) who
funded your project will need to understand what they paid for and be
able to justify the expenditure to their management. If your work is
successful and generates interest in im`aplementing the results by the
sponsor or in continuing the line of research, managers will be tapped
again for funds, and a later project team, who are not necessarily
experts at the outset, will need to use your documentation to come up
to speed. A clearly written Work Statement, along with your final
report, can be used by them as their starting points.

\bibliographystyle{plain}
\nocite{*}
\bibliography{biblioWS}

\end{document}
