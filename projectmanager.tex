\documentclass[12pt]{article}
\usepackage{listings}
\usepackage[colorlinks=true,pagebackref,linkcolor=blue]{hyperref}
\textwidth=7in
\textheight=9.5in
\topmargin=-1in
\headheight=0in
\headsep=.5in
\hoffset  -.85in

\lstset{
basicstyle=\footnotesize\ttfamily,
language=bash,
upquote=true,
breakatwhitespace=true,
columns=fullflexible,
keepspaces,
%numbers=none,
tabsize=3,
frame=blrt,
framextopmargin=5pt,
showstringspaces=false,
extendedchars=true
}

\pagestyle{empty}

\renewcommand{\thefootnote}{\fnsymbol{footnote}}

\begin{document}



\begin{center}
{\bf AMS 550.400 \quad Roles of Team Members\quad  Due Date:  {Monday,  Oct 8}}\\
\vskip.2in
{\footnotesize Last Compiled on \today}
\end{center}

\setlength{\unitlength}{1in}

\begin{picture}(6,.1) 
\put(0,0) {\line(1,0){6.25}}         
\end{picture}

\renewcommand{\arraystretch}{2}

\begin{center}
{\bf  Project Teams}
\end{center}

The following team assignment is final:
\begin{itemize}
\item Zhenhan Zhao and Shihong Li,
\item Yen Theng Tan and Joyce Tan, 
\item Jonathan Ho, Jordan Mandel and Yue Wu,  
\item Shannon Cebron, Michael Weinberger and Zhendan Zhu,
\item Ahmed Aly, Steven Su and Danni Tang, 
\item Huinan Zhang, Rong Fan and Jing Huang, 
\item Xiaohan Yang, Hua Hua and Du Yu,
\item Zichan Wang, Tianyu Luo and Yixuan Da.
\end{itemize}
\vskip0.25in

\begin{center}
    {\bf Duties of Project Manager}
\end{center}

\begin{itemize}
    \item Oversee the day to day activities of the project, keeping it
        productively focused,
    \item Work to keep the project on track and on time,
    \item Set up regular team meetings and attend all meetings with participants,
    \item Maintain excellent communication within the group and with
      the instructor,
    \item Manage by consensus when possible and make decision by authority
        when required, seeking the advice from instructor,
    \item Ensure that the team is informed of all important deadlines, dates, and meetings
    \item Assess individual strengths of the team members and make assignments
        on the project accordingly, with advice from the instructor,
    \item Function as intermediary among the different team members as needed,
    \item Bring unsolved problems to the attention of the instructor,
    \item Detect internal problems in the group that impact the schedule and
        deliverables,
    \item Act promptly and proactively to solve the problems or seek
      help and guidance from the instructor.
\end{itemize}

\vskip0.25in

\begin{center}
    {\bf Duties of Student Participants}
\end{center}

\begin{itemize}
  \item Work with the project manager to ensure that project is running on
      schedule and is meeting the specifications of the work statement,
  \item Accept and complete assignments given by the project manager and the
      instructor,
  \item Attend all team meetings,
  \item Keep the project manager apprised of any problems, issues, and
      absences,
  \item Bring any problems that arise to the attention of the project manager,
  \item Notify the project manager of any personal absences,
  \item Give unstintingly of his or her talents to make the project a success.
\end{itemize}

\vskip0.25in

\begin{center}
    {\bf Project Manger as a Problem Solver: Job Description}
\end{center}

Project management is a challenging and rewarding role. Not everybody can or
even wants to do it. Here are some thoughts about project management
that will help you decide whether you would like to be a project manager. If
you are selected, here is what will be expected of you.

Intense short-term (say nine-week) projects are too difficult for an
individual to accomplish in the
allotted time, so teamwork is essential for a successful project. The project
manager is the person who is in charge of maintaining coordination of a
project. It is a big responsibility. Let’s put it in perspective.


Your remaining nine-week period will be over almost before you know it, and in the
final days your team will have the difficult job of completing your 
project, preparing the final presentation, the Final Report
and wrapping up software and documentation. So it is very important that your
team be well-coordinated and working efficiently towards completion of the
project, else chaos rules and your team will suffer for it.

One of the first things your project team will do is write a work statement
that explains how your team will accomplish the work proposed in your
sponsor's project proposal. The work statement is very much like a contract
between you and your sponsor, detailing the project milestones, deliverables
and schedule, and you will negotiate its final form with the person appointed
by your sponsor as the sponsoring mentor, whose job it will be to monitor your
progress for the sponsor and serve as a discipline expert in the technical
area of the project. (Sometimes, a sponsor will appoint more than one
sponsoring mentor.

The success of a project depends in large part on several factors, including:
\begin{itemize}
    \item how well the work statement and its objectives are matched to the skills of the individual team members
    \item how effectively the disparate activities of the team members are coordinated towards achieving the objectives of the work statement
    \item how effectively the team is guided towards completing work on schedule
    \item how well the team efforts are communicated to the sponsoring mentor
    \item how well the team responds to the suggestions of the faculty mentor and sponsoring mentor.
\end{itemize}

These factors are summed up by the word ``organization.'' Good organization does
not often arise spontaneously. Usually it is brought about by selecting an
individual to take responsibility for coordinating and scheduling work and
maintaining external communications on behalf of the project. That person is
the project manager.


Here are some of the things a project manager must do, enumerated to underscore that a project manager has many responsibilities:
\begin{itemize}
    \item accept responsibility for the coordination of a team,
    \item work to promote a sense of shared enterprise and cooperation among the team members,
    \item understand the distinct skills of each team member and how each one may fit well within the project,
    \item seek and promote consensus among team members in defining project objectives,
    \item be fair minded in apportioning responsibilities to carry out the tasks needed to accomplish the objectives,
    \item gain acceptance of responsibility of each team member in performing assigned tasks,
    \item set goals and milestones to assure progress towards scheduled tasks,
    \item pay meticulous attention to the details of a project,
    \item take timely action when it becomes clear that milestones are not being met, assess cause and take corrective action,
    \item negotiate alternative objectives with sponsor if/when it becomes clear that a prior objective will fail,
    \item be prepared to make hard decisions when unresolved personnel problems arise,
    \item maintain continuing good communications with the sponsoring mentor and faculty mentor,
    \item watch the calendar to ensure timely delivery of all promised work products.
\end{itemize}


These are not the kinds of things that everyone can do or will want to do.
But, for the kind of person who wants to take on the extra responsibility, the
special satisfaction that comes with keeping a project running smoothly and on
track is itself a precious reward that makes the effort worthwhile. And don’t
forget: It is a rare opportunity for an undergraduate student or recent
graduate to manage a sponsored team of researchers—a successful performance in
that position will look good on your résumé.



\bibliographystyle{plain}
\nocite{*}
\bibliography{biblioMP}

\end{document}
